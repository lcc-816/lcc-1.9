\documentstyle{article}

\begin{document}
\begin{titlepage}
\normalsize
\vspace*{.7in}
\begin{center}
\begin{tabular}{c}
\bf\Large A Code Generation Interface for ANSI C \\[.5in]

Christopher W. Fraser \\
\em AT\&T Bell Laboratories, 600 Mountain Avenue, \\
\em Murray Hill, NJ 07974 \\[1ex]
and \\[1ex]
David R. Hanson \\
\em Department of Computer Science, Princeton University, \\
\em Princeton, NJ 08544 \\[.7in]

Research Report CS-TR-270-90 \\[1ex]
July 1990 \\
Last Revised September 1992 \\[.5in]

\end{tabular}
\end{center}

\begin{abstract}
\normalsize
\verb|lcc| is a retargetable, production compiler for ANSI~C;
it has been ported to the VAX, Motorola 68020, SPARC, and MIPS R3000,
and some versions have been in use for over two years.
It is smaller and faster than generally available alternatives,
and its local code is comparable.
This report describes the interface between the target-independent front end
and the target-dependent back ends.
The interface consists of shared data structures, a few functions,
and a dag language. While this approach couples the front
and back ends tightly, it results in efficient, compact compilers.
The interface is illustrated by detailing a complete code generator
that emits naive VAX code.
\end{abstract}

\end{titlepage}

%\setcounter{secnumdepth}{0}
%\bibliographystyle{abbrv}

\section{Introduction}

\verb|lcc| is a retargetable compiler for ANSI~C~\cite{ansi:Cstandard}.
It has been ported to the VAX, Motorola 68020, SPARC, and MIPS R3000.
It emits code that is comparable with that from other generally
available C compilers, but it runs up to twice as fast and is about
half the size~\cite{fraser:hanson:91b}.
\verb|lcc| is in production use at Princeton University
and AT\&T Bell Laboratories.

This paper describes the interface between the target-independent
front end and the target-dependent back ends.\footnote{References~\cite{fraser:hanson:91a}
and~\cite{fraser:hanson:92} were drawn from earlier versions of this report.
This report is slightly more detailed and corresponds to the latest
release of \verb|lcc|.}
Good code-generation interfaces are hard to design.
An inadequate interface may force each back end
to do work that could otherwise be done once in the front end.
Annotating frequently referenced variables for assignment
to registers is an example.
If the interface is too small, it
may not encode enough information to exploit new
machines thoroughly.  If the interface is too large, the back ends
may be needlessly complicated.
These competing demands
require careful engineering, and re-engineering as new targets expose
flaws. This paper reports the results of such experience.

The interface is illustrated with a {\em sample} code generator
for the VAX.  This code generator emits naive code; i.e., it uses only
the `RISC subset' of the VAX instruction set.  It is nevertheless
complete:  when used with a conforming preprocessor and library, the
compiler with this code generator passes the conformance section of
Version 2.00 of the Plum-Hall Validation Suite for ANSI~C, except that
it does not detect overflow in floating constants.
The production versions of \verb|lcc| use the interface described here,
but use back ends in which instruction selection and optimization are
generated automatically from a compact specification~\cite{fraser:sigplan89}.%
\footnote{The {\tt lcc} front end and a sample code generator are available
for anonymous {\tt ftp} from {\tt princeton.edu}.
The file {\tt README} in the directory {\tt pub/lcc} gives details.}

Unlike other interfaces, \verb|lcc|'s interface is not a monolithic
intermediate language~\cite{tanenbaum:toplas:82}. Such interfaces
promote decoupling between the front and back ends, but sacrifice
compiler performance~\cite{tanenbaum:sigplan:89}. With \verb|lcc|'s
interface, the front and back ends are tightly coupled; this approach
yields efficient, compact compilers, but can complicate maintenance
because changes to the front end may affect the back ends.  This
complication is less important for standardized languages like ANSI~C
because there will be few changes to the language.

The interface consists of a few shared data structures, 19 functions,
most of which are very simple, and a 36-operator dag language, which
encodes the executable code from a source program. The dag language
corresponds to the `intermediate language' used in other compilers, but
it is smaller than typical intermediate languages.
The functions, which can be implemented as true functions or as macros,
are listed in Appendix~\ref{appendix:interface} and are
described in the sections below.

The front and back ends are clients of each other.  The front end calls
on the back end to generate and emit code.  The back end calls on the
front end to perform output, allocate storage, interrogate types, and
manage nodes, symbols, and strings. These front-end services are
performed by functions that are summarized in Appendix~\ref{appendix:functions}.


\section{Configuration}

\label{configuration}
Target-specific configuration parameters specify the widths and
alignments of the basic datatypes and optionally define conditional
compilation flags.  They are defined in a `header' file, \verb|config.h|,
which is included when a target-specific \verb|lcc| is compiled.

The sample type metrics
and conditional compilation flags are defined as follows.
Target-specific components of the shared data structures
are defined in Sections~\ref{symbols} and \ref{dags}.
\begin{verbatim}
#define VAX

/* type metrics: size,alignment,constants */
#define CHAR_METRICS     1,1,0
#define SHORT_METRICS    2,2,0
#define INT_METRICS      4,4,0
#define FLOAT_METRICS    4,4,1
#define DOUBLE_METRICS   8,4,1
#define POINTER_METRICS  4,4,0
#define STRUCT_ALIGN     1

#define LEFT_TO_RIGHT    /* evaluate args left-to-right */
#define LITTLE_ENDIAN    /* right-to-left bit fields */
#define JUMP_ON_RETURN  0
\end{verbatim}

Type metrics are triples that give the size and alignment of the type
in bytes, along with a flag indicating whether or not constants of that
type can appear in instructions. A \verb|1| indicates that the
constants {\em cannot} appear in instructions; the front end generates
variables to hold them. The size of a type must be a multiple of its alignment.

Both the size and alignment for characters must be \verb|1|.
Unsigned and long integers are assumed to have metrics
identical to integer, and long doubles are assumed to have metrics
identical to double. The front end correctly treats all of these types
as separate, however.
\verb|POINTER_METRICS| apply to all pointer types, and pointers
must fit in unsigned integers.
The alignment of a structure is the maximum of the alignments
of its fields and \verb|STRUCT_ALIGN|.

If \verb|JUMP_ON_RETURN| is non-zero, the front end generates a jump to
a generated label for each \verb|return| statement and defines this
label at the end of each function.  Similar action is taken if the
compiler is asked to generate data for a debugger, because some
debuggers assume a common exit point.  Since the VAX does returns with
one instruction, \verb|JUMP_ON_RETURN| is defined as \verb|0|.

By default, the front end generates code that evaluates function
arguments from right to left;
defining \verb|LEFT_TO_RIGHT| yields the opposite order.
A sequence of bit fields is laid out from left to right
in one or more unsigned values; defining \verb|LITTLE_ENDIAN|
yields a right-to-left layout.
Defining \verb|LITTLE_ENDIAN| for the sample
makes the code compatible with existing VAX C compilers.
The standard permits either argument evaluation
or bit-field layout order, however.

The front end contains a few target-specific operations.
These are protected by conditional compilation on \verb|VAX|,
\verb|MIPS|, \verb|MC|, \verb|SPARC|, etc. Thus, \verb|VAX| is defined above.

\subsection{{\tt progbeg} and \tt progend}

\label{progbeg}\label{progend}
During initialization, the front end calls
\begin{verbatim}
void progbeg(int argc, char *argv[])
\end{verbatim}
\verb|argv[0..argc-1]| point
to those program arguments that are not recognized by the front end,
e.g., target-specific options.
Typical implementations of \verb|progbeg| process such
options and initialize themselves.
At the end of compilation, \verb|progend(void)| is called to give
the back end an opportunity to finalize its output.

The sample \verb|progbeg| initializes a register usage mask
\begin{verbatim}
void progbeg(int argc, char *argv[]) {
   rmask = ((~0)<<12)|1;
}
\end{verbatim}
which is described in Section~\ref{dags:registers}.
Finalization is unnecessary in the sample, so \verb|progend| is an empty macro:
\verb|#define progend(x)|.
All back-end functions except \verb|address| and \verb|gen| return nothing.
Those that are implemented as macros instead
of as true functions are defined in the configuration file.


\section{Symbols}

\label{symbols}
The front and back ends share two major data structures: symbol table
entries and dag nodes. Dag nodes are described in Section~\ref{dags}.
Symbol table entries are used for variables, constants, and labels.
They are represented by pointers to the following structure.
\begin{verbatim}
typedef struct symbol *Symbol;
struct symbol {        /* symbol table entries: */
   Xsymbol x;             /* type extension for code generator */
   char *name;            /* name */
   unsigned char scope;   /* scope level */
   unsigned char sclass;  /* storage class */
   unsigned defined:1;    /* 1 if defined */
   unsigned temporary:1;  /* 1 if a temporary */
   unsigned generated:1;  /* 1 if a generated identifier */
   unsigned addressed:1;  /* 1 if its address is taken */
   unsigned structarg:1;  /* 1 if parameter is a struct */
   Type type;             /* data type */
   union {
      struct {		  /* labels: */
	 int label;          /* label number */
      } l;
      struct {            /* constants: */
         Value v;            /* value */
         Symbol loc;         /* out-of-line location */
      } c;
      int seg;            /* globals, statics: definition segment */
   } u;
};
\end{verbatim}
The \verb|scope|, \verb|sclass|, and \verb|type| fields give the symbol's
scope level, its storage class, and its type, respectively.
Most of the bit fields flag self-explanatory attributes for each symbol;
fields relevant only to the front end are elided above.

Scope values classify symbols as constants, labels, or variables.
For labels, constants, and some variables,
a field of the union \verb|u| supplies additional data.

Labels have a \verb|scope| equal to the enumeration constant \verb|LABELS|.
The \verb|u.l.label| field is the numeric value of the label,
and \verb|name| is the string representation of that value.
Labels have no \verb|type| or \verb|sclass|.

Constants have a scope equal to \verb|CONSTANTS|, a \verb|sclass| equal
to \verb|STATIC|, and a \verb|name|
equal to the string representation of the constant.
The actual value of the constant is stored in the \verb|u.c.v| field,
which is defined by
\begin{verbatim}
typedef union value {   /* constant values: */
   char sc;                /* signed */
   short ss;               /* signed */
   int i;                  /* signed */
   unsigned char uc;
   unsigned short us;
   unsigned int u;
   float f;
   double d;
   char *p;                /* pointer to anything */
} Value;
\end{verbatim}
If a variable is generated to hold the constant,
\verb|u.c.loc| points to the symbol-table entry for that variable.

Variables have a \verb|scope| equal to \verb|GLOBAL|, \verb|PARAM|,
or \verb|LOCAL|$+k$ for nesting level $k$.
\verb|sclass| is \verb|STATIC|, \verb|AUTO|, \verb|EXTERN|, or \verb|REGISTER|.
The \verb|name| of most variables is the name used in the source code.
For temporaries and other generated variables, \verb|name| is a digit sequence.
\verb|temporary| and \verb|generated| are set for temporaries,
and \verb|generated| is set for labels and other generated variables, e.g., those
that hold constants.
\verb|structarg| is described in Section~\ref{emit}.
For global and static variables,
\verb|u.seg| gives the logical segment in which the variable is defined
(see Section~\ref{segment}).

The \verb|type| field for constants and variables points to
a structure that describes types. The \verb|size| and \verb|align|
fields of this structure give, respectively, the size and alignment constraints
of the type in bytes.

The \verb|x| field is an `extension' in which the back end stores
target-specific data for the symbol.
The sample has only two fields:
\begin{verbatim}
typedef struct {
   char *name;    /* name for back end */
   int offset;    /* frame offset */
} Xsymbol;
\end{verbatim}
This definition appears in the configuration file.
\verb|p->name| identifies the symbol to the front end, but the back end
may need to emit a different `name'. For example,
the `name' for locals is usually an offset from a frame pointer.
\verb|p->x.name| is the back end's name for the symbol.
For parameters and locals, \verb|p->x.offset| is the
offset from the frame pointer (see Sections~\ref{function} and~\ref{local}).

\subsection{\tt defsymbol}

\label{defsymbol}
Whenever the front end defines a new symbol with
\verb|scope| \verb|CONSTANTS|, \verb|LABELS|, or \verb|GLOBAL|
or a static variable, it calls
\verb|defsymbol(Symbol p)| to give the back end an opportunity to initialize
its \verb|Xsymbol| fields. The sample's \verb|defsymbol| is
\begin{verbatim}
void defsymbol(Symbol p) {
   if (p->scope == CONSTANTS)
      p->x.name = p->name;
   else if (p->scope >= LOCAL && p->sclass == STATIC)
      p->x.name = stringf("L%d", genlabel(1));
   else if (p->generated)
      p->x.name = stringf("L%s", p->name);
   else
      p->x.name = stringf("_%s", p->name);
}
\end{verbatim}
The back end's name for a constant is just the constant itself.
For generated symbols including labels, \verb|x.name| is the front end's name,
which is a digit string, prefixed with an \verb|L|,
and an underscore prefixes the names of other symbols.
\verb|stringf| returns a pointer to a string formatted as specified
by its \verb|printf|-style arguments.

For \verb|scope| \verb|PARAM| and  \verb|LOCAL|$+k$,
the \verb|Xsymbol| fields are initialized
by \verb|function| (see Section~\ref{function})
and \verb|local| (see Section~\ref{local}), respectively,
and symbols that represent address computations
are initialized by \verb|address| (see Section~\ref{address}).

\subsection{{\tt import} and \tt export}

\label{import}\label{export}
A symbol can be exported or imported.
Non-static variables and functions are exported
in order to be available to other,
separately compiled modules. Likewise, variables and functions used
in one module, but defined in another one, are imported
in the former module.
The front end identifies an exported or imported symbol
by calling \verb|export(Symbol p)| or \verb|import(Symbol p)|. \verb|export|
is always called {\em before} the symbol is defined;
\verb|import|, however, may be called any time, before or after the symbol
is used.

The sample back end emits an assembler directive for \verb|export|,
but no output is required for \verb|import|:
\begin{verbatim}
#define export(p) print(".globl %s\n", (p)->x.name)
#define import(p)
\end{verbatim}
\verb|print| is a front-end function similar to the standard \verb|printf|.

\subsection{\tt segment}

\label{segment}
The front end manages four logical segments:
\verb|CODE|, \verb|BSS|, \verb|DATA|, and \verb|LIT|.
Executable code is emitted into the \verb|CODE| segment,
uninitialized variables are defined in the \verb|BSS| segment,
initialized variables are defined and initialized in the \verb|DATA| segment,
and constants appear in the \verb|LIT| segment.

The front end announces a segment change by calling
\verb|segment(int s)| where \verb|s| is one of the segments listed above.
\verb|segment| maps the logical segments onto the segments
provided by the target machine. The \verb|CODE| and \verb|LIT| segments
can be mapped to read-only segments; the others must be mapped to read/write segments.
The sample mapping is
\begin{verbatim}
void segment(int s) {
   switch (s) {
   case CODE: print(".text\n");   break;
   case  LIT: print(".text 1\n"); break;
   case DATA:
   case  BSS: print(".data\n");   break;
   }
}
\end{verbatim}

\subsection{\tt global}

\label{global}
\verb|global(Symbol p)| emits code to define a global variable.
The front end will have already directed the definition
to the appropriate logical segment by calling \verb|segment| and
set \verb|p->u.seg| to that segment,
and it will follow the call to \verb|global|
with any appropriate calls to the data initialization functions.
\verb|global| handles the necessary alignment adjustments and
the actual definition.

The sample definition for global \verb|y| is simply
\verb|y|'s \verb|x.name| field preceded by an alignment directive, if necessary:
\begin{verbatim}
void global(Symbol p) {
   switch (p->type->align) {
   case 2: print(".align 1; "); break;
   case 4: print(".align 2; "); break;
   case 8: print(".align 3; "); break;
   }
   print("%s:", p->x.name);
}
\end{verbatim}

\subsection{\tt defconst}

\label{defconst}

\verb|defconst(int ty, Value v)| emits the scalar \verb|v|.
\verb|ty| indicates which field of
\verb|v| is to be emitted according the following table.

\begin{center}
\begin{tabular}{lll}
\tt ty	& \tt v \it field	& \it type \\ \hline
\tt C	& \tt v.uc		& character \\
\tt S	& \tt v.us		& short \\
\tt I	& \tt v.i		& int \\
\tt U	& \tt v.u		& unsigned \\
\tt P	& \tt v.p		& any pointer type \\
\tt F	& \tt v.f		& float \\
\tt D	& \tt v.d		& double \\
\end{tabular}
\end{center}
The codes \verb|S|, \verb|I|, \ldots\ are identical to the type suffixes
used for the dag operators, which are described in Section~\ref{dags}.
The signed fields \verb|v.sc| and \verb|v.ss| can be used instead
of \verb|v.uc| and \verb|v.us|, but \verb|defconst|
must initialize only the specified number of bits.

The sample \verb|defconst| is
\begin{verbatim}
void defconst(int ty, Value v) {
   switch (ty) {
   case C: print(".byte %d\n",   v.uc); break;
   case S: print(".word %d\n",   v.us); break;
   case I: print(".long %d\n",   v.i ); break;
   case U: print(".long 0x%x\n", v.u ); break;
   case P: print(".long 0x%x\n", v.p ); break;
   case F:
      print(".long 0x%x\n", ((unsigned *) &v.f)[0]);
      break;
   case D: 
      print(".long 0x%x,0x%x\n", ((unsigned *) &v.d)[0],
         ((unsigned *) &v.d)[1]);
      break;
   }
}
\end{verbatim}
In the production compilers, \verb|defconst| accommodates cross-compilation,
so it corrects for different representations and byte orders.

If \verb|ty| is \verb|P|, \verb|v.p| holds a numeric constant of some pointer type.
These originate from declarations like \verb|char *p=(char *)0xF0|.
\verb|defaddress| emits addresses relative to a symbol.

Few ANSI~C compilers can leave the encoding of floating-point constants
to the assembler, because few assemblers can cope with the effect of
casts on these constants.  For example, the correct initialization for
\begin{verbatim}
double x = (float)0.3;
\end{verbatim}
is \verb|.long 0x999a3f99,0x0|.
The directive \verb|.double 0.3| would erroneously initialize \verb|x|
to the equivalent of
\begin{verbatim}
.long 0x99993f99,0x0999a9999
\end{verbatim}
because it cannot represent the effect of the cast.


\subsection{{\tt defstring}, {\tt defaddress}, and \tt space}

\label{defstring}
\label{defaddress}
\label{space}

\verb|defstring(int len, char *s)| emits code to initialize
a string of length \verb|len| to the characters in \verb|s|.
The front end converts escape sequences, like \verb|\n|, into
the corresponding ASCII characters.

\verb|defaddress(Symbol p)| emits the address denoted by \verb|p|.
\verb|space(int n)| emits code to allocate \verb|n| zero bytes.

The sample \verb|defstring|, \verb|defaddress|, and \verb|space| are
\begin{verbatim}
void defstring(int len, char *s) {
   while (len-- > 0)
      print(".byte %d\n", *s++);
}

#define defaddress(p) print(".long %s\n", (p)->x.name)

#define space(x) print(".space %d\n", (x))
\end{verbatim}


\section{Functions}

The front end completely consumes each function before passing any part
of the function to the back end.  This organization permits certain
optimizations.  For example, only by processing complete functions can
the front end identify the locals and parameters whose address is not taken; only
these variables may be assigned to registers.

\subsection{\tt function}

\label{function}

The front end accumulates functions into private data structures.  At
the end of each function, it calls \verb|function| to generate and emit
code. The typical form of \verb|function| is
\begin{flushleft}\tt
\verb|void function(Symbol f, Symbol caller[], Symbol callee[], int ncalls) {| \\
\ \ \ \ldots {\rm initialize} \\
\ \ \ gencode(caller, callee); \\
\ \ \ \ldots {\rm emit prologue} \\
\ \ \ emitcode(); \\
\ \ \ \ldots {\rm emit epilogue} \\
\verb|}|
\end{flushleft}
\verb|gencode| is a front-end routine that traverses the front end's
private structures and passes each dag to the back end's \verb|gen|
(see Section~\ref{gen}), which selects code, annotates the dag to
record its selection, and returns a dag pointer. \verb|emitcode| is a
front-end routine that traverses the private structures again and
passes each of the pointers from \verb|gen| to \verb|emit| (see
Section~\ref{emit}), which emits the code.

This organization offers the back end flexibility in
generating function prologue and epilogue code.
Before calling \verb|gencode|, \verb|function| initializes
the \verb|Xsymbol| fields of the function's parameters, as described
below, and does other per-function initializations, if necessary.
After calling \verb|gencode|, the size of the activation record,
or {\em frame}, the number of registers used, etc.\ are known;
this information is usually needed to emit the prologue.
After calling \verb|emitcode| to emit the code for the body of the function,
\verb|function| emits the epilogue.

The argument \verb|f| to \verb|function| is the pointer
to the symbol-table entry for the current function,
and \verb|ncalls| is the number of calls to other functions
made by the current function. \verb|ncalls| is useful
on targets like the SPARC where `leaf' functions get special treatment.

\verb|caller| and \verb|callee| are arrays of pointers to symbol
table entries; each is terminated with a zero pointer.
The symbols in \verb|caller| are the function parameters
as passed by a caller; those in \verb|callee| are the parameters
as seen within the function. For most functions,
the symbols in each array are the same, but they
can differ in both \verb|sclass| and \verb|type|.
For example, in
\begin{verbatim}
foo(x) float x; { ... }
\end{verbatim}
a call to \verb|foo| passes the actual argument as a double.
Within \verb|foo|, \verb|x| is a float.
Thus, \verb|caller[0]->type| refers to `double' and \verb|callee[0]->type|
refers to `float.' And in
\begin{verbatim}
int strlen(register char *s) { ... }
\end{verbatim}
\verb|caller[0]->sclass| is \verb|AUTO| and \verb|callee[0]->sclass| is \verb|REGISTER|.
Even without register declarations, the front end assigns
frequently referenced parameters to the \verb|REGISTER| class,
and \verb|callee|'s \verb|sclass| is set accordingly.
This assignment is made only when there are no explicit register {\em locals}
to avoid interfering with the programmer's intentions (see Section~\ref{local}).

\verb|caller| and \verb|callee| are passed to \verb|gencode|.
If \verb|caller[i]->type| is not equal to \verb|callee[i]->type|
or if \verb|caller[i]->sclass| is not equal to \verb|callee[i]->sclass|,
\verb|gencode| generates an assignment of \verb|caller[i]|
to \verb|callee[i]|. If the types are not equal, this assignment
may include a conversion; for example, the assignment to \verb|x| in
\verb|foo| includes a truncation of a double to a float.
For parameters that include register declarations, \verb|function|
must assign a register and initialize the \verb|x| field accordingly,
or change the \verb|callee|'s \verb|sclass| to \verb|AUTO|.

\verb|function| could also change \verb|callee[i]->sclass| from \verb|AUTO|
to \verb|REGISTER| if it wished to assign a register to that parameter.
On the MIPS, for example, some of the parameters are passed in registers,
so \verb|function| assigns those registers to the corresponding
\verb|callee|s in leaf functions. If, however, \verb|callee[i]->addressed| is set,
the address of the parameter is taken in the function body,
and it must be stored in memory on most machines.

Initialization of the \verb|Xsymbol| fields of the symbols
in \verb|caller| and \verb|callee| depends on the frame layout,
which is target specific. Figure~\ref{fig:frame} shows the layout
of the VAX frame. The stack grows towards
lower addresses and towards the top of the page.

\begin{figure}
\begin{center}
\setlength{\unitlength}{12pt}
\begin{picture}(10,16)
\thinlines

\put( 0,14){\makebox(10, 2){argument build area}}
	\put( 0,14){\line( 1, 0){10}}
	\put(13,15.5){\vector(-1,0){3}}\put(13,15){\makebox(1.5,1){\tt sp}}
\put( 0,10){\makebox(10, 4){locals}}
	\put( 0,10){\line( 1, 0){10}}
\put( 0, 9){\makebox(10, 1){\tt 0}}
	\put( 0, 9){\line( 1, 0){10}}
	\put(13, 9.5){\vector(-1,0){3}}\put(13, 9){\makebox(1.5,1){\tt fp}}
\put( 0, 8){\makebox(10, 1){\sc psw}}
	\put( 0, 8){\line( 1, 0){10}}
\put( 0, 7){\makebox(10, 1){previous \tt ap}}
	\put( 0, 7){\line( 1, 0){10}}
\put( 0, 6){\makebox(10, 1){previous \tt fp}}
	\put( 0, 6){\line( 1, 0){10}}
\put( 0, 5){\makebox(10, 1){previous \tt pc}}
	\put( 0, 5){\line( 1, 0){10}}
\put( 0, 3){\makebox(10, 2){saved registers}}
	\put( 0, 3){\line( 1, 0){10}}
\put( 0, 2){\makebox(10, 1){argument count}}
	\put( 0, 2){\line( 1, 0){10}}
	\put(13, 2.5){\vector(-1,0){3}}\put(13, 2){\makebox(1.5,1){\tt ap}}
\put( 0, 0){\makebox(10, 2){actual arguments}}

\thicklines
\put( 0, 0){\line( 1, 0){10}}
\put( 0, 0){\line( 0, 1){16}}
\put(10,16){\line( 0,-1){16}}
\put(10,16){\line(-1, 0){10}}

\end{picture}
\end{center}
\caption{VAX Frame Layout.\label{fig:frame}}
\end{figure}

Arguments are referenced by displacement-mode addressing
with positive offsets from register \verb|ap|, so
the first argument is at address \verb|4(ap)|. Locals are referenced
via negative offsets from \verb|fp|, e.g., the first local is at \verb|-4(fp)|.
The `argument build area' is used to store arguments to functions that are
called by the current function. The front end `un-nests' calls so that
the back end does not need to deal with nested calls. The argument build area
can thus be used for all calls and must be large enough to hold the largest
argument list. When a function is called, the caller's argument build area
becomes the callee's `actual arguments'.

Typical VAX calling sequences can handle nested calls,
so using an argument build area is not strictly necessary.
But other targets, such as the MIPS, require this approach,
so it's used here to illustrate the technique. This approach also
has the advantage that stack overflow can occur only at function
entry, which is useful on targets that require explicit
prologue code to detect stack overflow.

The sample version of \verb|function| is
\begin{verbatim}
static int framesize;     /* size of activation record */
static int offset;        /* current frame offset */
static int argbuildsize;  /* size of argument build area */

\end{verbatim}
\begin{verbatim}
void function(Symbol f, Symbol caller[],
   Symbol callee[], int ncalls) {
   int i;

   offset = 4;
   for (i = 0; caller[i] && callee[i]; i++) {
      offset = roundup(offset, caller[i]->type->align);
      callee[i]->x.offset = caller[i]->x.offset = offset;
      callee[i]->x.name = caller[i]->x.name = stringf("%d(ap)", offset);
      offset += caller[i]->type->size;
      callee[i]->sclass = AUTO;
   }
   usedmask = argbuildsize = framesize = offset = 0;
   gencode(caller, callee);
   print("%s:.word 0x%x\n", f->x.name, usedmask&~0x3f);
   framesize += 4*nregs + argbuildsize;
   print("subl2 $%d,sp\n", framesize);
   if (isstruct(freturn(f->type)))
      print("movl r1,-4(fp)\n");
   emitcode();
   if (glevel > 1)
      print("ret\n");
}
\end{verbatim}
The VAX \verb|calls| instruction saves the general registers specified
by the entry mask, \verb|ap|, \verb|fp|,
and the return address, \verb|pc|, as shown in the frame figure above.
\verb|function| computes the size of the locals and argument build area,
given by \verb|framesize| and \verb|argbuildsize|, respectively.

The first part of \verb|function| initializes the \verb|x.offset| and \verb|x.name| fields
of each \verb|caller| and \verb|callee| symbol to the appropriate offset and name, respectively.
The running \verb|offset| is rounded up to the alignment for each argument.
\verb|roundup(n,m)| is a front-end macro that returns \verb|n| rounded up to the
next multiple of \verb|m|. Depending on the type metrics, the size of an
argument may not be a multiple of longwords (e.g., 3-byte structures),
but the front end ensures that the minimum alignment for each \verb|caller[i]|
is that for integers, which keeps the VAX stack longword-aligned.
\verb|stringd(n)| is a front-end function that returns the string
representation of the integer \verb|n|.

This version of \verb|function| does not support register
declarations, so each \verb|callee|'s storage \verb|sclass| is
set to \verb|AUTO|.

During code generation, \verb|argbuildsize| is increased when
code for calls is generated (see Section~\ref{gen}), \verb|offset| is adjusted in response
to the definition of locals and block boundaries,
and \verb|framesize| records \verb|offset|'s maximum
(see Sections~\ref{local} and \ref{blockbeg}).
Before calling \verb|gencode|, \verb|function| clears
\verb|usedmask|, which records the registers used in the function body,
and clears \verb|argbuildsize|, \verb|framesize|, and \verb|offset|.

After \verb|gencode| returns, \verb|usedmask| and \verb|framesize|
hold the information needed to generate the prologue.
\verb|framesize| is adjusted to include the argument build area
and space for saving all of the registers.
This space, which is in addition to that specified by the register save mask,
is used to save registers across those instructions that
destroy fixed registers, e.g., \verb|movc3|.
The \verb|subl2| instruction allocates the remainder of the frame.

The last instruction of the prologue is emitted only for
functions that return structures. For a function type \verb|ty|,
\verb|freturn(ty)| gives the type of the value returned by the function,
and \verb|isstruct(ty)| is true if \verb|ty| is a structure or union type.
Section~\ref{dags:calls} gives details on returning structures.

If \verb|JUMP_ON_RETURN| had been defined as \verb|1| in the configuration,
returns would have been followed by a jump to the label
that follows the code emitted by \verb|emitcode|. If the front
end is passed the \verb|-g|$n$ option, it sets the global
variable \verb|glevel| to $n$
and behaves as if \verb|JUMP_ON_RETURN| is \verb|1|, so the code
above emits the necessary \verb|ret| instruction at the end
of each function.

\subsection{\tt local}

\label{local}

During the execution of \verb|gencode|, the front end announces local
variables by calling \verb|local(Symbol p)|, where \verb|p| points the
relevant symbol-table entry.  It announces temporaries likewise; these
have \verb|p->temporary| set. \verb|local| must initialize \verb|p|'s
\verb|Xsymbol| fields. That is, it must set \verb|p->x.offset| and \verb|p->x.name|
so that they identify a
stack offset or register number, depending on the availability of
registers and on the value of \verb|p->sclass|, which is \verb|AUTO| or
\verb|REGISTER|.

For each block, the front end first announces locals with explicit
register declarations, in order of declaration, to permit programmer
control of register assignment. Then it announces the rest, starting
with those that appear to be most frequently referenced.  It assigns
\verb|REGISTER| class to even these locals {\em if} their address
is never taken and if their estimated frequency of use exceeds two.
This announcement order and \verb|sclass| override collaborate to put the most
promising locals in registers even if no registers were declared.  As
with parameters, \verb|local| could assign a register to \verb|p| and
change \verb|p->sclass| from \verb|AUTO| to \verb|REGISTER|, but it
should do so only if \verb|p->addressed| is not set.

If \verb|p->sclass| is \verb|REGISTER|, \verb|local|
can decline to allocate a register by setting \verb|p->sclass| to \verb|AUTO|
and initializing \verb|p->x.offset| and \verb|p->x.name| to the appropriate frame offset
and address string, respectively.
This choice is illustrated by the sample version:
\begin{verbatim}
void local(Symbol p) {
   offset = roundup(offset + p->type->size, p->type->align);
   offset = roundup(offset, 4);
   p->x.offset = -offset;
   p->x.name = stringf("%d(fp)", -offset);
   p->sclass = AUTO;
}
\end{verbatim}
The second \verb|roundup| keeps \verb|offset| and hence the stack
aligned on longwords, as described above.

\subsection{\tt address}

\label{address}
The front end calls \verb|address(Symbol q, Symbol p, int n)| to
initialize \verb|q->x| to a symbol that represents an address of the form $x+{\tt n}$,
where $x$ is the address represented by \verb|p| and \verb|n| is positive or negative.
Like \verb|defsymbol|, \verb|address| initializes \verb|q->x|, but
does so based on the values of \verb|p->x| and \verb|n|.
The sample \verb|address| is
\begin{verbatim}
void address(Symbol q, Symbol p, int n) {
   if (p->scope == GLOBAL || p->sclass == STATIC || p->sclass == EXTERN)
      q->x.name = stringf("%s%s%d", p->x.name, n >= 0 ? "+" : "", n);
   else {
      q->x.offset = p->x.offset + n;
      q->x.name = stringf("%d(%s)", q->x.offset,
         p->scope == PARAM ? "ap" : "fp");
   }
}
\end{verbatim}
which computes \verb|q->x.offset| and \verb|q->x.name| for locals and parameters,
or sets \verb|q->x.name| to \verb|p->x.name| concatenated with \verb|+n| or \verb|-n|
for other variables. 

For example, in
\begin{verbatim}
struct node { struct node *link; int count; } a;
f() { int b[10]; b[4] = a.count; ... }
\end{verbatim}
suppose \verb|a| and \verb|b| point to the symbol-table entries
for \verb|a| and \verb|b|, respectively.
\verb|a->x.name| is set to \verb|"_a"| by \verb|defsymbol|, and
\verb|b->x.offset| and \verb|b->x.name| are set to, respectively,
\verb|-40| \verb|"-40(fp)"| by \verb|local|.
\verb|address(q1,a,4)| is called with \verb|q1| representing the address
of \verb|a.count|, and \verb|q1->x.name| is set to \verb|"_a+4"|.
Likewise, \verb|address(q2,b,16)| sets \verb|q2->x.offset|
and \verb|q2->x.name| to, respectively, \verb|-24| and \verb|"-24(fp)"|,
which together denote the address of \verb|b[4]|.

\verb|address| accepts globals, parameters, and locals,
and is called only after these symbols have been initialized
by {\tt defsymbol}, {\tt function}, or {\tt local}.

\subsection{{\tt blockbeg} and \tt blockend}

\label{blockbeg}\label{blockend}
Source-language blocks bracket the lifetime of locals.
\verb|gencode| announces the beginning and end of a block
by calling \verb|blockbeg(Env *e)| and \verb|blockend(Env *e)|, respectively.
\verb|Env| is target specific and typically includes
the data necessary to reuse that portion of the local frame
space associated with the block
and to release any registers assigned to locals within the block.
The sample \verb|Env| is
\begin{verbatim}
typedef struct {
   unsigned rmask;
   int offset;
} Env;
\end{verbatim}
The fields save the values of \verb|rmask| and \verb|offset| at the beginning
of a block so that they can be restored on the end of the block.
The sample \verb|blockbeg| and \verb|blockend| are thus

\noindent
\begin{minipage}[t]{.5\textwidth}
\vspace{\parskip}
\begin{verbatim}
void blockbeg(Env *e) {
   e->rmask = rmask;
   e->offset = offset;
}
\end{verbatim}
\vspace{\parskip}
\end{minipage}
\begin{minipage}[t]{.5\textwidth}
\vspace{\parskip}
\begin{verbatim}
void blockend(Env *e) {
   if (offset > framesize)
      framesize = offset;
   offset = e->offset;
   rmask = e->rmask;
}
\end{verbatim}
\vspace{\parskip}
\end{minipage}

\noindent
\verb|blockend| also updates \verb|framesize| if the locals for the current
block require more space than previous blocks.
The sample could do without the \verb|rmask| field, but
if its \verb|local| assigned registers to locals, it would need
the field to release those registers.

Temporaries --- locals with \verb|temporary| set --- to which
\verb|local| assigned registers live only for the expressions in which
they are used.  They are announced by \verb|local| as usual, but are
used only in the dags passed to next call on \verb|gen|
(see~\ref{gen}).  \verb|gen| can thus release all registers assigned to
temporaries.


\section{Dags}

\label{dags}
Executable code is specified by dags.  A function body is a sequence of
forests of dags, each of which is passed the back end via \verb|gen|,
as described below.  Dag nodes, or simply nodes, are defined by
\begin{verbatim}
typedef struct node *Node;
struct node {           /* dag nodes: */
   Opcode op;              /* operator */
   short count;            /* reference count */
   Symbol syms[MAXSYMS];   /* symbols */
   Node kids[MAXKIDS];     /* operands */
   Node link;              /* next dag in the forest */
   Xnode x;                /* back-end's type extension */
};
\end{verbatim}

The \verb|kids| point to the operand nodes.  Some operators also
take symbol table pointers as operands; these appear in the \verb|syms|
array.  The default and minimum allowable value for both \verb|MAXKIDS|
and \verb|MAXSYMS| is \verb|2|; larger values can be defined for the
back end's convenience in the configuration file. \verb|count|
holds the number of references to this node from \verb|kids| in other
nodes. \verb|link| points to the root of the next dag in the forest.

The \verb|x| field is the back end's `extension' to nodes.  The
configuration defines the type \verb|Xnode| to hold the per-node data
that the back end needs to generate code.  The sample \verb|Xnode| is
\begin{verbatim}
typedef struct {
   unsigned visited:1; /* 1 if dag has been linearized */
   int reg;            /* register number */
   unsigned rmask;     /* unshifted register mask */
   unsigned busy;      /* busy regs */
   int argoffset;      /* ARG: argument offset */
   Node next;          /* next node on emit list */
} Xnode;
\end{verbatim}
Section~\ref{gen} describes the fields.

The \verb|op| field holds an operator. The last character of each is a
{\em type suffix} from Table~\ref{dags:suffix-table}.
For example, the generic operator \verb|ADD| has the variants
\verb|ADDI|, \verb|ADDU|, \verb|ADDP|, \verb|ADDF|, and \verb|ADDD|.

\begin{table}
\begin{center}
\begin{tabular}{lll}
\em type suffix	& \em type \\ \hline
\tt C		& char \\
\tt S		& short \\
\tt I		& int \\
\tt U		& unsigned \\
\tt P		& any pointer type \\
\tt F		& float \\
\tt D		& double \\
\tt B		& structure or block \\
\tt V		& void \\
\end{tabular}
\end{center}
\caption{Type Suffixes.\label{dags:suffix-table}}
\end{table}

Table~\ref{dags:op-table} lists each generic operator, its valid type
suffixes, and the number of \verb|kids| and \verb|syms| that it uses;
multiple values for \verb|kids| indicate type-specific variations,
which are detailed below.  For most operators, the type suffix denotes
the type of operation to perform and the type of the result.
Exceptions are \verb|ADDP|, in which an integer operand
is added to a pointer operand, and \verb|SUBP|, which
subtracts the second integer operand from the first pointer operand.
The operators for assignment, comparison, arguments, and some calls return
no results; their type suffixes denote the type of operation to perform.

\begin{table}
\begin{center}
\begin{tabular}{lllll}
\tt syms & \tt kids & \it operator & \it type suffixes	& \it operation \\ \hline
 1	& 0	& \tt ADDRF	& \verb|    P    |	& address of a parameter \\
 1	& 0	& \tt ADDRG	& \verb|    P    |	& address of a global \\
 1	& 0	& \tt ADDRL	& \verb|    P    |	& address of a local\\
 1	& 0	& \tt CNST	& \verb|CSIUPFD  |	& constant \\[1.5ex]

	& 1	& \tt BCOM	& \verb|   U     |	& bitwise complement \\
	& 1	& \tt CVC	& \verb|  IU     |	& convert from \tt char \\
	& 1	& \tt CVD	& \verb|  I  F   |	& convert from \tt double \\
	& 1	& \tt CVF	& \verb|      D  |	& convert from \tt float \\
	& 1	& \tt CVI	& \verb|CS U  D  |	& convert from \tt int \\
	& 1	& \tt CVP	& \verb|   U     |	& convert from pointer \\
	& 1	& \tt CVS	& \verb|  IU     |	& convert from \tt short \\
	& 1	& \tt CVU	& \verb|CSI P    |	& convert from \tt unsigned \\
	& 1	& \tt INDIR	& \verb|CSI PFDB |	& fetch \\
	& 1	& \tt NEG	& \verb|  I  FD  |	& negation \\[1.5ex]

	& 2	& \tt ADD	& \verb|  IUPFD  |	& addition \\
	& 2	& \tt BAND	& \verb|   U     |	& bitwise \sc AND \\
	& 2	& \tt BOR	& \verb|   U     |	& bitwise inclusive \sc OR \\
	& 2	& \tt BXOR	& \verb|   U     |	& bitwise exclusive \sc OR \\
	& 2	& \tt DIV	& \verb|  IU FD  |	& division \\
	& 2	& \tt LSH	& \verb|  IU     |	& left shift \\
	& 2	& \tt MOD	& \verb|  IU     |	& modulus \\
	& 2	& \tt MUL	& \verb|  IU FD  |	& multiplication \\
	& 2	& \tt RSH	& \verb|  IU     |	& right shift \\
	& 2	& \tt SUB	& \verb|  IUPFD  |	& subtraction \\[1.5ex]

 2	& 2	& \tt ASGN	& \verb|CSI PFDB |	& assignment \\
 1	& 2	& \tt EQ	& \verb|  I  FD  |	& jump if equal \\
 1	& 2	& \tt GE	& \verb|  IU FD  |	& jump if greater than or equal \\
 1	& 2	& \tt GT	& \verb|  IU FD  |	& jump if greater than \\
 1	& 2	& \tt LE	& \verb|  IU FD  |	& jump if less than or equal \\
 1	& 2	& \tt LT	& \verb|  IU FD  |	& jump if less than \\
 1	& 2	& \tt NE	& \verb|  I  FD  |	& jump if not equal \\[1.5ex]

 2	& 1	& \tt ARG	& \verb|  I PFDB |	& argument \\
 1	& 1 2	& \tt CALL	& \verb|  I  FDBV|	& function call \\
	& 0 1	& \tt RET	& \verb|  I  FD V|	& return from function \\[1.5ex]

  	& 1	& \tt JUMP	& \verb|        V|	& unconditional jump \\
 1	& 0	& \tt LABEL	& \verb|        V|	& label definition \\
\end{tabular}
\end{center}
\caption{Node Operators.\label{dags:op-table}}
\end{table}

The leaf operators yield the address of a variable or the value of a
constant. \verb|syms[0]| identifies the variable or constant.

The unary operators accept and yield a number, except for \verb|INDIR|,
which accepts an address and yields the value at that address.
There is no \verb|BCOMI|; signed integers are complemented using
\verb|BCOMU|. The binary operators accept two numbers and yield one.

The type suffix for a conversion operator denotes the type of the
result.  For example, \verb|CVUI| converts an unsigned (\verb|U|) to an
signed integer (\verb|I|).  Conversions between unsigned and short and
between unsigned and character are unsigned conversions; those between
integer and short and between integer and character are signed
conversions.  For example, \verb|CVSU| converts an unsigned short to an
unsigned while \verb|CVSI| converts an signed short to a signed
integer.

The front end composes conversions to form those not in the table.
For example, it converts a short to a float by first
converting it to an int and then a double.  The 16
conversion operators are represented by arrows in Figure~\ref{fig:conversions}.
Composed conversions follow the path from the source
type to the destination type.

\begin{figure}
\begin{center}
\setlength{\unitlength}{7pt}
\begin{picture}(20,12)
\multiput(1,2)(6,0){3}{\begin{picture}(0,3)
	\put(0,0){\vector(0,1){3}}\put(0,0){\vector(0,-1){0}}\end{picture}}
\multiput(7,7)(6,0){2}{\begin{picture}(0,3)
	\put(0,0){\vector(0,1){3}}\put(0,0){\vector(0,-1){0}}\end{picture}}
\multiput(2,6)(6,0){3}{\begin{picture}(4,0)
	\put(0,0){\vector(1,0){4}}\put(0,0){\vector(-1,0){0}}\end{picture}}
\put( 1, 1){\makebox(0,0){\tt F}}
\put( 7, 1){\makebox(0,0){\tt S}}
\put(13, 1){\makebox(0,0){\tt S}}

\put( 1, 6){\makebox(0,0){\tt D}}
\put( 7, 6){\makebox(0,0){\tt I}}
\put(13, 6){\makebox(0,0){\tt U}}
\put(19, 6){\makebox(0,0){\tt P}}

\put( 7,11){\makebox(0,0){\tt C}}
\put(13,11){\makebox(0,0){\tt C}}
\end{picture}
\end{center}
\vspace{-1.5pc}\small\em
\caption{Conversions~\label{fig:conversions}}
\end{figure}

There is no \verb|CVUD|; conversion of an unsigned \verb|u| to a double
is done by the equivalent of the expression
\begin{verbatim}
(int)u >= 0 ? (double)(int)u : (double)(int)u + UINT_MAX + 1
\end{verbatim}
where \verb|UINT_MAX| is the ANSI-specified maximum value for an unsigned.
Likewise, there is no \verb|CVDU|; conversion of a double \verb|d| to an unsigned
is done by the equivalent of the expression
\begin{verbatim}
d >= INT_MAX + 1 ? (unsigned)(int)(d-(INT_MAX+1)) + INT_MAX + 1
                 : (unsigned)(int)d
\end{verbatim}
where \verb|INT_MAX| is the ANSI-specified maximum value for a signed integer.

\verb|ASGN| stores the value of \verb|kids[1]| in the cell addressed by
\verb|kids[0]|. \verb|syms[0]| and \verb|syms[1]| point to symbol table
entries for integer constants that give the size of the value and
its alignment, respectively. These are most useful for \verb|ASGNB|,
which implements structure assignment and other
`block moves' (e.g., initialization of automatic arrays).

For the comparisons, \verb|syms[0]| points to a symbol-table entry for
the label to jump to if the comparison is true.  \verb|JUMPV| is an
unconditional jump to the address computed by \verb|kids[0]|. \verb|LABEL|
defines the label given by \verb|syms[0]| and is otherwise a no-op.
As on most computers, signed comparisons are used for unsigned equals and not equals. 

Function calls have a \verb|CALL| node preceded by zero or more
\verb|ARG| nodes. The front end `un-nests'
function calls, so \verb|ARG| nodes are always associated with the next
\verb|CALL| node in the forest. The order of the \verb|ARG| nodes is
right-to-left unless the configuration parameter \verb|LEFT_TO_RIGHT|
is defined, which it is for the sample.
A \verb|CALL| node's \verb|syms[0]| points to a symbol
whose only non-null field is \verb|type|,
which is the function type of the callee.

\verb|ARG| nodes establish the value computed by \verb|kids[0]| as the
next argument.  \verb|syms[0]| and \verb|syms[1]| point to symbol table
entries for integer constants that give the size of the argument and
its alignment, respectively.

\label{dags:calls}

In \verb|CALL| nodes, \verb|kids[0]| computes the address of the callee.
\verb|CALLB| calls functions that return structures;
\verb|kids[1]| computes the address of a
temporary local variable to hold the returned value. There is no
\verb|RETB|; the front end uses a \verb|RETV| preceded by an
\verb|ASGNB| to the structure addressed by the first local. The
\verb|CALLB| code and the function prologue must collaborate to store the
\verb|CALLB|'s \verb|kids[1]| into the callee's first local.
\verb|function| (Section~\ref{function}), \verb|local| (Section~\ref{local})
and the code emitted below for \verb|CALLB| (Section~\ref{emit})
illustrate such collaboration.
\verb|CALLB| nodes have a \verb|count| of \verb|0| because the front end references
the temporary wherever the returned value is referenced.

In \verb|RET| nodes, \verb|kids[0]| computes the value returned, except
for \verb|RETV| nodes, which are childless.

Character and short integer arguments are always promoted to the
corresponding integer type even in the presence of a prototype.
The promoted values are converted back to the intended
type upon entry to the function. The front end accomplishes
this conversion by specifying the intended types for the callee
as described in Section~\ref{function}. 
For example, the body for
\begin{verbatim}
f(char c) { f(c); }
\end{verbatim}
becomes two forests, which are linearized below;
the \verb|syms| columns list the \verb|x.name| fields.
\begin{verbatim}
 #  op      count kids    syms
 1. ADDRFP   1            4(ap)
 2. ADDRFP   1            4(ap)
 3. INDIRI   1     2                                    
 4. CVIC     1     3                                    
 5. ASGNC    0     1  4                                 
                                  
 1. ADDRFP   1            4(ap)
 2. INDIRC   1     1                                    
 3. CVCI     1     2                                    
 4. ARGI     0     3      4 4                           
 5. ADDRGP   1            _f                             
 6. CALLI    0     5      4                             
\end{verbatim}
The first forest holds one dag, which converts the actual argument to the
intended type.  The second forest holds two dags.
The first (nodes 1--4) promotes \verb|c| to pass it as an integer,
and the second (nodes 5 \& 6) calls \verb|f|.

Unsigned variants of \verb|ASGN|, \verb|INDIR|, \verb|ARG|,
\verb|CALL|, and \verb|RET| were omitted as unnecessary.  Signed and
unsigned integers have the same size, so the corresponding signed
operator is used instead.  Likewise, there is no \verb|CALLP| or
\verb|RETP|. A pointer is returned by using \verb|CVPU| and
\verb|RETI|. A pointer-valued function is called by using \verb|CALLI|
and \verb|CVUP|.

In Table~\ref{dags:op-table}, the operators listed at and following \verb|ASGN|
are used for their side-effects. They always appear as roots in the
forest of dags, and they appear in the order in which they must be executed.
Except for \verb|CALLD|, \verb|CALLF| and \verb|CALLI|, their reference
counts are always zero.  Even these \verb|CALL| nodes have zero
reference counts when their values are unused.

\subsection{\tt gen}

\label{gen}
\label{dags:registers}
The front end calls \verb|gen| to select code.  It passes \verb|gen| a
forest of dags.  For example,
\begin{verbatim}
int i, *p; f() { i = *p++; }
\end{verbatim}
yields the forest linearized below.
\begin{verbatim}
 #  op      count kids      syms
 1. ADDRGP   2              _p
 2. INDIRP   2     1
 3. CNSTI    1              4
 4. ADDP     1     2  3
 5. ASGNP    0     1  4
 6. ADDRGP   1              _i
 7. INDIRI   1     2
 8. ASGNI    0     6  7
\end{verbatim}
This forest consists of three dags, rooted at nodes 2, 5, and 8 above.
The \verb|INDIRP| node, which fetches the value of \verb|p|, comes
before node 5, which changes \verb|p|, so the original value of
\verb|p| is available for subsequent use by node 7, which fetches the
integer pointed to by that value.

\verb|gen| traverses the forest and selects code, but it
emits nothing because it may be necessary to determine, for example,
the registers needed before the function prologue
can be emitted (see Section~\ref{function}).
So \verb|gen| merely annotates the nodes to identify the code selected,
and it returns a pointer that is ultimately passed to the back end's
\verb|emit| to actually output the code.  Once the front end calls
\verb|gen|, it does not inspect the contents of the nodes again, so
\verb|gen| may modify them freely.

The sample code generator emits naive code, so \verb|gen| concerns
itself mainly with register allocation. 
The sample \verb|gen|
\begin{verbatim}
Node gen(Node p) {
   Node head, *last;

   for (last = &head; p; p = p->link)
      last = linearize(p, last, 0);
   for (p = head; p; p = p->x.next) {
      ralloc(p);
      if (p->count == 0 && sets(p))
         putreg(p);
   }
   return head;
}
\end{verbatim}
linearizes each dag in the forest, in execution order, and then
allocates registers for each node.  If the node sets a register, but no
subsequent node references it, which is indicated by a reference
\verb|count| of \verb|0|, \verb|gen| releases the register.  Finally,
it returns the linearized node list for traversal by \verb|emit|.

\verb|gen| linearizes the dags before allocating registers to simplify
the insertion of spills and reloads. The function
\begin{verbatim}
static Node *linearize(Node p, Node *last, Node next) {
   if (p && !p->x.visited) {
      last = linearize(p->kids[0], last, 0);
      last = linearize(p->kids[1], last, 0);
      p->x.visited = 1;
      *last = p;
      last = &p->x.next;
   }
   *last = next;
   return last;
}
\end{verbatim}
linearizes the dag at \verb|p| and inserts it into the growing list of
nodes between \verb|*last| and \verb|next|.

\verb|ralloc| handles register allocation for a single node.
\begin{verbatim}
static void ralloc(Node p) {
   int i;

   switch (generic(p->op)) {
   case ARG:
      argoffset = roundup(argoffset, p->syms[1]->u.c.v.i);
      p->x.argoffset = argoffset;
      argoffset += p->syms[0]->u.c.v.i;
      if (argoffset > argbuildsize)
         argbuildsize = roundup(argoffset, 4);
      break;
   case CALL:
      argoffset = 0;
      break;
   }
   for (i = 0; i < MAXKIDS; i++)
      putreg(p->kids[i]);
   p->x.busy = rmask;
   if (needsreg(p))
      getreg(p);
}
\end{verbatim}
\verb|generic| strips the type suffix from an operator and
thus simplifies the switch.  \verb|ralloc| calls \verb|putreg| to
release the registers allocated for the node's children and then it
calls \verb|getreg| to allocate a register for the node itself if it needs one.  It sets
the node's \verb|x.busy| to record the busy registers; \verb|emit|
needs this information for a few operators.  The register state is
encoded in \verb|rmask|; \verb|rmask&(1<<r)| is \verb|1| if register
\verb|r| is busy, for \verb|r| from \verb|0| to \verb|11|.

\verb|CALL| and \verb|ARG| nodes require extra steps.  \verb|ARG| nodes
produce no result; instead, they store the value computed by
\verb|kids[0]| into the next location in the argument build area.
\verb|syms[0]| and \verb|syms[1]| point to constant symbols that give
the size and alignment of the argument.  \verb|argoffset| is the
running offset into the argument build area.
\verb|argoffset| is rounded up to the appropriate alignment boundary, saved
in the node's \verb|x.argoffset| for use in \verb|emit|, and incremented
by the size of the argument.  \verb|argbuildsize| tracks the maximum
\verb|argoffset| needed by the current function.  The case for \verb|CALL| nodes
clears \verb|argoffset| for the next set of \verb|ARG| nodes.

\verb|getreg|
accepts a node and allocates a register for it:
\begin{verbatim}
static void getreg(Node p) {
   int r, m = optype(p->op) == D ? 3 : 1;

   for (r = 0; r < nregs; r++)
      if ((rmask&(m<<r)) == 0) {
         p->x.rmask = m;
         p->x.reg = r;
         rmask |= sets(p);
         usedmask |= sets(p);
         return;
      }
   r = spillee(p, m);
   spill(r, m, p);
   getreg(p);
}
\end{verbatim}
\verb|optype(op)| returns the type suffix of operator \verb|op|.
\verb|m| is set to the mask \verb|1| if
the result of \verb|p->op| needs an ordinary register and \verb|3| if it
needs a double register.
\verb|getreg| loops over the registers. If it finds one that's free,
it sets the node's \verb|x.reg| field to the register allocated and the
\verb|x.rmask| field to the mask. It also
updates \verb|usedmask|, which is used to generate the prologue
described in Section~\ref{function}; \verb|sets(p)| returns
\verb|p->x.rmask<<p->x.reg|.  If no registers are free, \verb|getreg|
spills a register and calls itself recursively to try again.
Section~\ref{spills} describes spills.

\verb|putreg| releases registers:
\begin{verbatim}
static void putreg(Node p) {
   if (p && --p->count <= 0)
      rmask &= ~sets(p);
}
\end{verbatim}
Dags can use result registers multiple times, so \verb|putreg|
decrements the reference count and frees the register only when the
last reference is removed by clearing the appropriate bits in
\verb|rmask|.

\subsection{\tt emit}

\label{emit}
\verb|emit| emits the linearized forest. The sample walks down the list,
switches on the opcode to identify the code to emit:
\begin{verbatim}
void emit(Node p) {
   for (; p; p = p->x.next) {
      Node a = p->kids[0], b = p->kids[1];
      int r = p->x.reg;
      switch (p->op) {
      case CNSTC: case CNSTI: case CNSTP:
      case CNSTS: case CNSTU:
         print("movl $%s,r%d\n", p->syms[0]->x.name, r);
         break;
      case ADDRGP: case ADDRFP: case ADDRLP:
         print("moval %s,r%d\n", p->syms[0]->x.name, r);
         break;
      ...
      }
   }
}
\end{verbatim}
The individual cases emit naive code for a single operator.
The \verb|CNST| cases above emit a VAX instruction that loads a constant
into a register. \verb|defsymbol| stored the constant string in
\verb|p->syms[0]->x.name|, and \verb|ralloc| stored the result register
name \verb|p->x.reg|. The \verb|ADDR| cases above emit a VAX
instruction that moves the address of a variable into the result
register.

Most of the unary operators share a common pattern, which the sample
abstracts into a macro:
\begin{verbatim}
#define suffix(p)    ".fdbwllll."[optype((p)->op)]
#define unary(inst)  print("%s%c r%d,r%d\n", inst, suffix(p), a->x.reg, r)
case BCOMU:                             unary("mcom" );  break;
case NEGD:  case NEGF:  case NEGI:      unary("mneg" );  break;
case CVCI:                              unary("cvtb" );  break;
case CVCU:                              unary("movzb");  break;
case CVSI:                              unary("cvtw" );  break;
case CVSU:                              unary("movzw");  break;
case CVDF:  case CVDI:                  unary("cvtd" );  break;
case CVFD:                              unary("cvtf" );  break;
case CVUC:  case CVUS:                  unary("cvtl" );  break;
case CVIC:  case CVIS:  case CVID:      unary("cvtl" );  break;
case CVIU:  case CVUI:                  unary("mov"  );  break;
case CVPU:  case CVUP:                  unary("mov"  );  break;
\end{verbatim}
\verb|unary| emits the VAX operator, the VAX type suffix --- which is
computed by indexing a string of such suffixes --- and the source and
destination registers. Most of the binary operators
\begin{verbatim}
#define binary(inst) print("%s%c3 r%d,r%d,r%d\n", inst, suffix(p), \
                        b->x.reg, a->x.reg, r)
case BANDU:                             binary("bic");   break;
case BORU:                              binary("bis");   break;
case BXORU:                             binary("xor");   break;
case ADDD:  case ADDF:                  binary("add");   break;
case ADDI:  case ADDP:  case ADDU:      binary("add");   break;
case SUBD:  case SUBF:                  binary("sub");   break;
case SUBI:  case SUBP:  case SUBU:      binary("sub");   break;
case MULD:  case MULF:                  binary("mul");   break;
case MULI:  case MULU:                  binary("mul");   break;
case DIVD:  case DIVF:  case DIVI:      binary("div");   break;
\end{verbatim}
and all of the comparisons
\begin{verbatim}
#define compare(cp)  print("cmp%c r%d,r%d; j%s %s\n", suffix(p), \
                        a->x.reg, b->x.reg, cp, p->syms[0]->x.name)
case EQD:   case EQF:   case EQI:       compare("eql" ); break;
case GED:   case GEF:   case GEI:       compare("geq" ); break;
case GEU:                               compare("gequ"); break;
case GTD:   case GTF:   case GTI:       compare("gtr" ); break;
case GTU:                               compare("gtru"); break;
case LED:   case LEF:   case LEI:       compare("leq" ); break;
case LEU:                               compare("lequ"); break;
case LTD:   case LTF:   case LTI:       compare("lss" ); break;
case LTU:                               compare("lssu"); break;
case NED:   case NEF:   case NEI:       compare("neq" ); break;
\end{verbatim}
are handled similarly.

The cases
\begin{verbatim}
case INDIRC: case INDIRD: case INDIRF: case INDIRI:
case INDIRP: case INDIRS:
   print("mov%c (r%d),r%d\n", suffix(p), a->x.reg, r);
   break;
case ASGNC: case ASGND: case ASGNF: case ASGNI: case ASGNP: case ASGNS:
   print("mov%c r%d,(r%d)\n", suffix(p), b->x.reg, a->x.reg);
   break;
case JUMPV:
   print("jmp (r%d)\n", a->x.reg);
   break;
case LABELV:
   print("%s:", p->syms[0]->x.name);
   break;
\end{verbatim}
emit code to indirectly load and store memory cells, to jump indirectly,
and to emit a label definition.

The cases
\begin{verbatim}
case ARGD: case ARGF: case ARGI: case ARGP:
   print("mov%c r%d,%d(sp)\n", suffix(p),
      a->x.reg, p->x.argoffset);
   break;
case CALLD: case CALLF: case CALLI: case CALLV:
   save(p->x.busy&0x3e);
   print("calls $0,(r%d)\n", a->x.reg);
   if (p->op != CALLV)
      print("mov%c r0,r%d\n", suffix(p), r);
   restore(p->x.busy&0x3e);
   break;
case RETD: case RETF: case RETI:
   print("mov%c r%d,r0; ret\n", suffix(p), a->x.reg);
   break;
case RETV:
   print("ret\n");
   break;
\end{verbatim}
emit code to move an argument onto the stack, to call a procedure, and
to return a value. The \verb|ARG| cases use \verb|x.argoffset|,
into which \verb|ralloc| stored the stack offset for the argument.
Procedures may destroy registers 1--5, so the \verb|CALL| cases use
\begin{verbatim}
static void save(unsigned mask) {
   int i;

   for (i = 1; i < nregs; i++)
      if (mask&(1<<i))
         print("movl r%d,%d(fp)\n", i,
            4*i - framesize + argbuildsize);
}
\end{verbatim}
to emit code to save any of those registers that are busy;
\verb|restore| is similar.
The \verb|RET| cases merely copy any return value into register 0 and return.

Several binary operators require special handling. The shift
instructions use a syntax that differs slightly from the other binary instructions:
\begin{verbatim}
case RSHI: case LSHI: case LSHU:
   print("ashl r%d,r%d,r%d\n", b->x.reg, a->x.reg, r);
   break;
\end{verbatim}
\verb|RSHI| and \verb|LSHI| generate the same code because the front
end negates the shift count for \verb|RSHI| if the configuration
parameter \verb|VAX| is defined. No instructions directly
implement unsigned right shift, division, or modulus, so \verb|emit|
uses a field-extraction instruction for the first and library calls for the others:
\begin{verbatim}
case RSHU:
   print("subl3 r%d,$32,r0; extzv r%d,r0,r%d,r%d\n",
      b->x.reg, b->x.reg, a->x.reg, r);
   break;
case DIVU:
   save(p->x.busy&0x3e);
   print("pushl r%d; pushl r%d; calls $2,udiv; movl r0,r%d\n",
      b->x.reg, a->x.reg, r);
   restore(p->x.busy&0x3e);
   break;
case MODU:
   save(p->x.busy&0x3e);
   print("pushl r%d; pushl r%d; calls $2,urem; movl r0,r%d\n",
      b->x.reg, a->x.reg, r);
   restore(p->x.busy&0x3e);
   break;
\end{verbatim}
Finally, ANSI C requires that \verb|a%b| equal \verb|a-(a/b)*b|,
so \verb|MODI| computes it just this way:
\begin{verbatim}
case MODI:
   print("divl3 r%d,r%d,r0; mull2 r%d,r0; subl3 r0,r%d,r%d\n",
      b->x.reg, a->x.reg, b->x.reg, a->x.reg, r);
   break;
\end{verbatim}

Only the structure instructions remain:
\begin{verbatim}
case INDIRB:
   print("moval (r%d),r%d\n", a->x.reg, r);
   break;
case ASGNB:
   save(p->x.busy&0x3f);
   print("movc3 $%s,(r%d),(r%d)\n", p->syms[0]->x.name,
      b->x.reg, a->x.reg);
   restore(p->x.busy&0x3f);
   break;
case ARGB:
   save(p->x.busy&0x3f);
   print("movc3 $%s,(r%d),%d(sp)\n", p->syms[0]->x.name,
      a->x.reg, p->x.argoffset);
   restore(p->x.busy&0x3f);
   break;
\end{verbatim}
The scalar \verb|INDIR|s and \verb|ASGN|s load and store values
directly, but structures won't fit in registers, so their instructions
manipulate {\em addresses} instead. \verb|ASGNB| uses a
block move instruction, which needs the size from
\verb|p->syms[0]->x.name|; it also destroys registers 0--5, so
\verb|emit| arranges to save and restore their values. \verb|ARGB|
operates similarly, but copies the structure into the stack instead.
Finally,
\begin{verbatim}
case CALLB:
   save(p->x.busy&0x3e);
   if (a->x.reg == 1) {
      print("movl r1,r0\n");
      a->x.reg = 0;
   }
   if (b->x.reg != 1)
      print("movl r%d,r1\n", b->x.reg);
   print("calls $0,(r%d)\n", a->x.reg);
   restore(p->x.busy&0x3e);
   break;
\end{verbatim}
is like an ordinary call, but it also passes the address at which to
store the return value in register 1; if the address of the {\em function}
is already in register 1, it is first moved out of the way into
register 0, which is not otherwise allocated.

If \verb|NOARGB| is defined (e.g., in the configuration file),
the front end uses Sun's convention for passing structures by value.
It builds dags that copy structure arguments to temporaries,
passes pointers to these temporaries, adds an extra indirection
to references to these parameters in the callee, and changes
the types of the callee's formals to reflect this convention.
It also sets \verb|structarg| for these parameters to distinquish
them from bona fide structure pointer parameters.

\subsection{\tt asmcode}

\label{asmcode}
\verb|asm("...")| directives in C source programs cause
\verb|emitcode| to call \verb|asmcode|:
\begin{verbatim}
void asmcode(char *str, Symbol argv[]) {
   for ( ; *str; str++)
      if (*str == '%' && str[1] >= 0 && str[1] <= 9)
         print("%s", argv[*++str]->x.name);
      else
         print("%c", *str);
   print("\n");
}
\end{verbatim}
\verb|str| points to the string given in the \verb|asm| directive,
and \verb|argv| is an array of symbol table pointers.

The front end replaces occurrences of \verb|%x|
that appear in the \verb|asm| string
with \verb|%|$k$, where $k$ is a 1-byte integer from \verb|0| to \verb|9|.
\verb|argv[|$k$\verb|]| is the symbol-table entry for identifier \verb|x|.
\verb|asmcode| emits \verb|str|, replacing \verb|%|$k$ with appropriate
back-end names.
For example, the \verb|asm| in
\begin{verbatim}
int x; f(register i) { int j; asm("addl3 %i,%j,%x"); ... }
\end{verbatim}
becomes \verb|addl3 4(ap),-4(fp),_x|.

\section{Spills}

\label{spills}
When \verb|getreg| finds that all registers are busy,
one or more must be {\em spilled} now and {\em reloaded}
when the values are needed again.
Handling spills correctly is difficult and a common
source of compiler bugs. Test cases that expose
bugs in the spill code are necessarily complex.
As such, and for completeness,
the implementation described in this section is adapted
from that used in the production versions of \verb|lcc|.
Reference~\cite{fraser:hanson:92} describes the important
differences and explains the rationale behind this spiller.

The dag constructed by the front end minimizes the reevaluation of
common subexpressions. Spills are, in a sense,
a result of the front end's eagerness to avoid
reevaluation, and handling spills amounts to
`breaking the dags' generated by the front end.
A node representing a common subexpression is changed
so that it stores the value in a temporary, and
subsequent references to that node are edited
to load the value from the temporary.

Spilling involves three major steps:
Identifying the registers to spill,
generating the code for the spills at the correct
location the output stream, and
generating the code for the reloads, again at the correct locations.
These steps are performed at the end of \verb|getreg|:
\begin{verbatim}
static void getreg(Node p) {
   int r, m = optype(p->op) == D ? 3 : 1;
   ...
   r = spillee(p, m);
   spill(r, m, p);
   getreg(p);
}
\end{verbatim}
\verb|spillee| identifies the register (\verb|m==1|) or register pair (\verb|m==3|)
to be spilled on behalf of \verb|p|,
and \verb|spill| generates the spill and reload code.
Once \verb|r| has been spilled,
it is available,
and the call to \verb|getreg| is guaranteed to succeed.

\verb|ralloc| frees registers as soon as possible.
If the available registers are exhausted,
it is because there are multiple references
to the nodes holding the registers, which arise
from common subexpressions and from multiple assignment, augmented
assignment, and the operators \verb|++| and \verb|--|.
Consider the following program.
\begin{verbatim}
double a[10],b[10];
int i;
f(){ i = (a[i]+b[i])*(a[i]-b[i]); }
\end{verbatim}
The initial linearized forest is shown to the left of
the vertical line in the display below.
\begin{flushleft}\tt
\begin{tabular}{rlllc|cll}
\#\ & op      & kids    & syms  & count &count   & uses   & sets  \\
 1. & ADDRGP  &         & \_i   &   2   &   1    &        & r1    \\
 2. & INDIRI  &  1      &       &   1   &   0    & r1     & r2    \\
 3. & CNSTI   &         & 3     &   1   &   0    &        & r3    \\
 4. & LSHI    &  2  3   &       &   2   &   0    & r2 r3  & r2    \\
 5. & ADDRGP  &         & \_a   &   1   &   0    &        & r3    \\
 6. & ADDP    &  4  5   &       &   1   &   0    & r2 r3  & r3    \\
 7. & INDIRD  &  6      &       &   2   &   2    & r3     & r3 r4 \\
 8. & ADDRGP  &         & \_b   &   1   &   0    &        & r5    \\
 9. & ADDP    &  4  8   &       &   1   &   0    & r2 r5  & r2    \\
10. & INDIRD  &  9      &       &   2   &   2    & r2    \\
11. & ADDD    &  7 10   &       &   1   &   1    & r3 r4 \\
12. & SUBD    &  7 10   &       &   1   &   1    & r3 r4 \\
13. & MULD    & 11 12   &       &   1   &   1 \\
14. & CVDI    & 13      &       &   1   &   1 \\
15. & ASGNI   &  1 14   &       &   0   &   0    & r1 \\
\end{tabular}
\end{flushleft}
Nodes with \verb|count| fields greater than 1 represent
four common subexpressions: the lvalue of \verb|i| (node 1),
the addressing expression \verb|i<<3| (node 4), and
the rvalues of \verb|a[i]| (node 7) and \verb|b[i]| (node 10).
If only registers 1--5 are available, \verb|ralloc| runs
out of registers at node 10.
The linearized forest at that point is shown to the right of
the vertical line in the display above.
As indicated by the non-zero \verb|count|s and the \verb|sets| column,
node 1 is using \verb|r1| and
node 7 is using \verb|r3| and \verb|r4|.
Node 10 needs a register pair,
but only registers \verb|r2| and \verb|r5| are available.
Note that the \verb|count| of node 4, which is \verb|i<<3|,
has dropped to 0 because \verb|ralloc| has
processed both uses (nodes 6 and 9).

\verb|getreg| calls \verb|spillee| to identify a register pair
to be spilled so that it can be used for node 10.
\verb|spillee| chooses the register whose next use
is the most distant in the linearized forest.
This choice is analogous to the optimal demand paging strategy that
replaces pages whose next use is most distant
in the execution stream~\cite{freiburghouse}.

For each register,
\verb|spillee| simply searches down the linearized forest
for register uses and records the most distant.
\begin{verbatim}
static int spillee(Node dot, unsigned m) {
   int bestdist = -1, bestreg = 0, dist, r;
   Node q;

   for (r = 1; r < nregs - (m>>1); r++) {
      dist = 0;
      for (q = dot->x.next; q && !(uses(q)&(m<<r)); q = q->x.next)
         dist++;
      if (dist > bestdist) {
         bestdist = dist;
         bestreg = r;
      }
   }
   return bestreg;
}
\end{verbatim}
\verb|spillee| is called with \verb|dot| equal to the node at which
the spill is required, which is node 10 in the example above.
\verb|uses(p)| returns a bit mask giving the registers used by node \verb|p|,
i.e., the registers set by \verb|p|'s kids:
\begin{verbatim}
static unsigned uses(Node p) {
   int i;
   unsigned m = 0;

   for (i = 0; i < MAXKIDS; i++)
      if (p->kids[i])
         m |= sets(p->kids[i]);
   return m;
}
\end{verbatim}

In this example, \verb|spillee| finds \verb|r1| used
in node 15, which is at `distance' 4, and
\verb|r3| and \verb|r4| used in node 11 at distance 0.
Consequently, \verb|spillee| returns \verb|r1|, which denotes
both \verb|r1| and \verb|r2| because \verb|m| is 3.

The implementations of \verb|spillee| above and \verb|spill| below
assume that the instruction
emitted for each node reads the registers it uses before setting
the register allocated to it. This assumption
permits \verb|ralloc| to reassign registers used by a node to that node.
It also permits \verb|spillee| to start its scan {\em after} the
current instruction.
This assumption is invalid on machines with two-address instructions.

Actually spilling the register chosen by \verb|spillee| and inserting
the reloads is done by \verb|spill| and its associates, \verb|genspill|
and \verb|genreloads|.
In the example, these functions collaborate to `break the dag'
at node 1, which sets register \verb|r1| (\verb|r2| is already free).
\verb|genspill| generates a temporary and the nodes necessary
to store the value computed by node 1 into the temporary.
These new nodes are stitched into the linearized forest immediately
after node 1. \verb|genreloads| changes future {\em uses} of the
value computed by node 1 to reload the value from
the temporary instead of referencing node 1 directly.
The effect is to remove the common subexpression
by assigning it to a temporary.

\verb|spill| identifies the locations at which
to insert the spills by searching the linearized forest beyond \verb|dot|
for nodes that use the registers that are to be spilled, e.g.,
\verb|r1| in the example.
\begin{verbatim}
static void spill(int r, unsigned m, Node dot) {
   int i;
   Node p = dot;

   while (p = p->x.next)
      for (i = 0; i < MAXKIDS; i++)
         if (p->kids[i] && sets(p->kids[i])&(m<<r)) {
            Symbol temp = genspill(p->kids[i]);
            rmask &= ~sets(p->kids[i]);
            genreloads(dot, p->kids[i], temp);
         }
}
\end{verbatim}
\verb|genspill| allocates the temporary,
generates the spill code, and inserts it into the linearized forest
right after the node that set the spillee.
It returns the symbol-table entry for the temporary
so that it can be used by \verb|genreloads| to generate the reloads.
\begin{verbatim}
static Symbol genspill(Node p) {
   Symbol temp = newtemp(AUTO, typecode(p));
   Node q = p->x.next;

   linearize(newnode(ASGN + typecode(p),
      newnode(ADDRLP, 0, 0, temp), p, 0),
      &p->x.next, p->x.next);
   rmask &= ~1;
   for (p = p->x.next; p != q; p = p->x.next)
      ralloc(p);
   rmask |= 1;
   return temp;
}
\end{verbatim}
\verb|typecode| is like \verb|optype|, but maps \verb|U| to \verb|I|
because the front-end uses \verb|ASGNI| and \verb|INDIRI|
to store and load unsigned values. It also maps \verb|B| to
\verb|P| because registers always point structures.
The front-end's \verb|newtemp(sclass, type)|
allocates a new temporary or reuses an existing one if
its storage \verb|sclass| and \verb|type| code match those requested,
and it announces a new temporary by calling \verb|local| as usual.

The spill code is an \verb|ADDRLP| node to compute the
address of the temporary and an \verb|ASGN| node to store the value
into that address. The right operand of the \verb|ASGN| is simply
\verb|p|, the node that set the registers to spill, e.g.,
node 1 above.
A node is allocated and initialized by the front-end function
\verb|newnode(op,l,r,sym)|, where
\verb|op| is the operator, \verb|l| and \verb|r|
are \verb|kids[0]| and \verb|kids[1]|, and \verb|sym| is the symbol
table pointer for leaf nodes. \verb|newnode| also increments
the reference counts of \verb|l| and \verb|r|.

Finally, \verb|genspill| linearizes the spill code and
inserts it right after \verb|p|, which sets the spilled register.
The manipulation of \verb|rmask| ensures that \verb|ralloc|
assigns register \verb|r0| --- which is not otherwise allocated ---
to the \verb|ADDRLP| node to compute the address.
The linearized forest after \verb|genspill| returns is
\begin{verbatim}
 #  op      kids    syms  count  uses   sets
 1. ADDRGP          _i      1           r1
16. ADDRLP          -4(fp)  0           r0
17. ASGNP   16  1           0    r0 r1        
 2. INDIRI   1              0    r1     r2
...     
\end{verbatim}
Nodes 16 and 17 are the spill code. This code references
node 1, so its \verb|count| goes to 2 momentarily
until \verb|ralloc| processes the reference from the spill code.
The remaining reference is from node 15.
Node 16's symbol \verb|-4(fp)| is the temporary location.

After \verb|genspill| returns the temporary,
\verb|spill| frees the registers set by the node and calls \verb|genreloads|.
\verb|genreloads| searches the linearized forest
after \verb|dot| for uses of \verb|p|.
\begin{verbatim}
static void genreloads(Node dot, Node p, Symbol temp) {
   int i;
   Node last;

   for (last = dot; dot = dot->x.next; last = dot)
      for (i = 0; i < MAXKIDS; i++)
         if (dot->kids[i] == p) {
            dot->kids[i] = newnode(INDIR + typecode(p),
               newnode(ADDRL+P, 0, 0, temp), 0, 0);
            dot->kids[i]->count = 1;
            p->count--;
            linearize(dot->kids[i], &last->x.next, last->x.next);
            last = dot->kids[i];
         }
}
\end{verbatim}
Each use of \verb|p| is changed
to a reload of the temporary \verb|temp|.
The reload code is an \verb|ADDRLP| node to compute the
address of the temporary and an \verb|INDIR| node to load the value
from that address. There is only one reference to each reload, so the
\verb|INDIR|'s \verb|count| is set accordingly,
and \verb|p->count| is decremented to reflect the change.
The reload code is linearized and inserted into the linearized node
list immediately {\em before} its use.
For the example above, a reload is placed before node 15.
Since the reload is beyond the point that \verb|gen| has reached,
registers are allocated for these nodes by subsequent calls to \verb|ralloc|.

The linearized forest after \verb|spill| returns is
\begin{verbatim}
 #  op      kids    syms  count  uses   sets
 1. ADDRGP          _i      0           r1
16. ADDRLP          -4(fp)  0           r0
17. ASGNP   16  1           0    r0 r1        
 2. INDIRI   1              0    r1     r2
...
14. CVDI    13              1                 
18. ADDRLP          -4(fp)  1                 
19. INDIRP  18              1                 
15. ASGNI   19 14           0                 
\end{verbatim}
Nodes 18 and 19 are the reload.
Note that node 1's \verb|count| has become 0;
after inserting the reload, there are no unprocessed references.

Another spill occurs at node 11, which computes \verb|a[i]+b[i]|.
Registers \verb|r1| and \verb|r2|, which are set by node 10, are spilled,
and node 12 is edited to refer to the second temporary, which is reloaded
by nodes 22 and 23 inserted before node 12.
The last spill occurs at node 23, which is the reload
created for the second spill. \verb|r1| and \verb|r2|, which are
set by node 11, are spilled again, and node 13 is edited
to refer to the third temporary.
The final linearized forest and corresponding VAX code follows;
\verb|count| fields are omitted because they're all 0,
and each instruction shows which registers are used and set by each node.
\begin{verbatim}
VAX instruction     #  op      kids    syms
moval _i,r1         1. ADDRGP          _i
moval -4(fp),r0    16. ADDRLP          -4(fp)
movl r1,(r0)       17. ASGNP   16  1
movl (r1),r2        2. INDIRI   1
movl $3,r3          3. CNSTI           3
ashl r3,r2,r2       4. LSHI     2  3
moval _a,r3         5. ADDRGP          _a
addl3 r3,r2,r3      6. ADDP     4  5
movd (r3),r3        7. INDIRD   6
moval _b,r5         8. ADDRGP          _b
addl3 r5,r2,r2      9. ADDP     4  8
movd (r2),r1       10. INDIRD   9
moval -12(fp),r0   20. ADDRLP          -12(fp)
movd r1,(r0)       21. ASGND   20 10
addd3 r1,r3,r1     11. ADDD     7 10
moval -20(fp),r0   24. ADDRLP          -20(fp)
movd r1,(r0)       25. ASGND   24 11
moval -12(fp),r5   22. ADDRLP          -12(fp)
movd (r5),r1       23. INDIRD  22
subd3 r1,r3,r1     12. SUBD     7 23
moval -20(fp),r3   26. ADDRLP          -20(fp)
movd (r3),r3       27. INDIRD  26
muld3 r1,r3,r1     13. MULD    27 12
cvtdl r1,r1        14. CVDI    13
moval -4(fp),r2    18. ADDRLP          -4(fp)
movl (r2),r2       19. INDIRP  18
movl r1,(r2)       15. ASGNI   19 14
\end{verbatim}
Spills have added nodes 16--27.


\section{Discussion}

\verb|lcc|'s code generation interface is smaller than most
because it omits the inessential and makes simplifying assumptions.
These omissions and assumptions do, however,
limit the interface's applicability to other languages and machines.

The datatype assumptions detailed in Section~\ref{configuration},
e.g., that unsigneds, integers, and long integers all have the same size,
make it possible to have only 9 type suffixes and 109 type-specific operators.
Relaxing these assumptions would increase this vocabulary;
e.g., adding a suffix for long doubles would also add at least
19 more operators.

The interface assumes that all pointer types have the same representation,
which precludes its use for word-addressed machines.
Differentiating between character and word pointers
would add another suffix and at least 13 more operators.

The operator repertoire omits some operators whose effect
can be synthesized from simpler ones.
For example, bit fields are accessed with shifting
and masking instead of specific bit-field operators, so
code quality may suffer on machines with bit-field instructions.
The front end special-cases one-bit fields
and generates efficient masking dags,
which often yields better code than code
that uses bit-field instructions.

The front end implements switch statements completely.
It generates a binary search of dense branch tables;
inline comparisons replace small tables~\cite{bernstein85}.
It fabricates the tables and indirect jumps
using the functions \verb|global| and \verb|defaddress|
and the \verb|JUMPV| operator.
This decision prevents back ends from using larger `case' instructions,
which usually combine a bounds check and an indirect branch through a table,
but these instructions are increasingly rare, and some don't suit C anyway.
For example, the branch table for the VAX's \verb|casel| instruction is limited
to 16-bit offsets.

The interface has no direct support for building
a flow graph and other structures that facilitate
global optimization. More elaborate versions of
\verb|function| and \verb|gen| could collaborate
to build the relevant structures, perform optimizations,
and invoke the simpler \verb|gen|.
The front end's \verb|gencode| and \verb|emitcode| traverse
an approximation of a flow graph; future work may focus on
a machine-independent optimizer that edits
this graph while preserving the current interface.

To date, the interface has been used only for ANSI~C.
But it has little that is specific to C, and it could be used for
similar languages and perhaps for languages with features like nested
procedures, objects, and exception handling.
Existing compilers for some object-oriented languages with these features,
such as C\verb|++|, Modula-3, and Eiffel, generate C, so, in principle,
the interface could be used for these languages.

%\bibliography{refs,lib}
\begin{thebibliography}{1}

\bibitem{ansi:Cstandard}
American National Standard Institute, Inc., New York.
\newblock {\em American National Standards for Information Systems, Programming
  Language C {ANSI} X3.159--1989}, 1990.

\bibitem{bernstein85}
R.~L. Bernstein.
\newblock Producing good code for the case statement.
\newblock {\em Software---Practice \& Experience}, 15(10):1021--1024, Oct.
  1985.

\bibitem{fraser:sigplan89}
C.~W. Fraser.
\newblock A language for writing code generators.
\newblock {\em Proceedings of the SIGPLAN'89 Conference on Programming Language
  Design and Implementation, SIGPLAN Notices}, 24(7):238--245, July 1989.

\bibitem{fraser:hanson:91a}
C.~W. Fraser and D.~R. Hanson.
\newblock A code generation interface for {ANSI C}.
\newblock {\em Software---Practice \& Experience}, 21(9):963--988, Sept. 1991.

\bibitem{fraser:hanson:91b}
C.~W. Fraser and D.~R. Hanson.
\newblock A retargetable compiler for {ANSI C}.
\newblock {\em SIGPLAN Notices}, 26(10):29--43, Oct. 1991.

\bibitem{fraser:hanson:92}
C.~W. Fraser and D.~R. Hanson.
\newblock Simple register spilling in a retargetable compiler.
\newblock {\em Software---Practice \& Experience}, 22(1):85--99, Jan. 1992.

\bibitem{freiburghouse}
R.~A. Freiburghouse.
\newblock Register allocation via usage counts.
\newblock {\em Communications of the ACM}, 17(11):638--642, Nov. 1974.

\bibitem{tanenbaum:sigplan:89}
A.~S. Tanenbaum, M.~F. Kaashoek, K.~G. Langendoen, and C.~J.~H. Jacobs.
\newblock The design of very fast portable compilers.
\newblock {\em SIGPLAN Notices}, 24(11):125--131, Nov. 1989.

\bibitem{tanenbaum:toplas:82}
A.~S. Tanenbaum, H.~van Staveren, and J.~W. Stevenson.
\newblock Using peephole optimization on intermediate code.
\newblock {\em ACM Transactions on Programming Languages and Systems},
  4(1):21--36, Jan. 1982.

\end{thebibliography}

\appendix

\section{Interface Functions}
\label{appendix:interface}

In the table below, \dag\ flags those interface functions that are implemented by
macros in the sample.

\begin{center}
\begin{tabular}{cl}
\em Section	& \em Interface Function \\ \hline
\ref{address}	& \tt void address(Symbol, Symbol, int) \\
\ref{asmcode}	& \tt void asmcode(char *, Symbol []) \\
\ref{blockbeg}	& \tt void blockbeg(Env *e) \\
\ref{blockend}	& \tt void blockend(Env *e) \\
\makebox[0pt][r]{\dag\ }%
\ref{defaddress}& \tt void defaddress(Symbol) \\
\ref{defconst}	& \tt void defconst(int, Value) \\
\ref{defstring}	& \tt void defstring(int, char *) \\
\ref{defsymbol}	& \tt void defsymbol(Symbol) \\
\ref{emit}	& \tt void emit(Node) \\
\makebox[0pt][r]{\dag\ }%
\ref{export}	& \tt void export(Symbol) \\
\ref{function}	& \tt void function(Symbol, Symbol [], Symbol [], int) \\
\ref{gen}	& \tt Node gen(Node) \\
\ref{global}	& \tt void global(Symbol) \\
\makebox[0pt][r]{\dag\ }%
\ref{import}	& \tt void import(Symbol) \\
\ref{local}	& \tt void local(Symbol) \\
\ref{progbeg}	& \tt void progbeg(int, char *[]) \\
\makebox[0pt][r]{\dag\ }%
\ref{progend}	& \tt void progend(void) \\
\ref{segment}	& \tt void segment(int) \\
\makebox[0pt][r]{\dag\ }%
\ref{space}	& \tt void space(int) \\
\end{tabular}
\end{center}


\section{Front-End Functions}

\label{appendix:functions}

\begin{description}

\item[{\tt void *alloc(int n)}] permanently allocates \verb|n| bytes and returns
a pointer to the first byte.  No alignment is guaranteed.

\item[{\tt void fprint(int fd, char *fmt, \ldots)}] prints up to 5 arguments on
the file descriptor \verb|fd|. See \verb|print| for formatting details.
If \verb|fd| is not \verb|1| (standard output),
\verb|fprint| flushes the output buffer for \verb|fd| via \verb|outflush|.

\item[{\tt Type freturn(Type t)}] is the type of the return value for
function type \verb|t|.

\item[{\tt int generic(int op)}] is the generic version
of the type-specific operator \verb|op|.

\item[{\tt int genlabel(int n)}] increments the generated identifier
counter by \verb|n| and returns its previous value.

\item[{\tt int is\it type\/\tt(Type t)}] are a set of type predicates that
return non-zero if type \verb|t| is the type in the following table.

\begin{center}
\begin{tabular}{lllll}
\it predicate	& \it type	&& \it predicate	& \it type \\ \hline
\tt isarith	& arithmetic	&& \tt isint		& integral \\
\tt isarray	& array		&& \tt isptr		& pointer \\
\tt ischar	& character	&& \tt isscalar		& scalar \\
\tt isdouble	& double	&& \tt isstruct		& structure or union \\
\tt isenum	& enumeration	&& \tt isunion		& union \\
\tt isfloat	& floating	&& \tt isunsigned	& unsigned \\
\tt isfunc	& function	&& \\
\end{tabular}
\end{center}

\item[{\tt Node newnode(int op, Node l, Node r, Symbol sym)}] allocates a node
and initializes the \verb|op| field to \verb|op|, \verb|kids[0]| and \verb|kids[1]|
to \verb|l| and \verb|r|, and \verb|syms[0]| to \verb|sym| and returns
a pointer to the new node. If \verb|l| and \verb|r| are non-null,
their \verb|count| fields are incremented.

\item[{\tt Symbol newconst(Value v, int t)}] installs an constant with value \verb|v|
and type suffix \verb|t| into the symbol table, if necessary,
and returns a pointer to the symbol-table entry.

\item[{\tt Symbol newtemp(int sclass, int t)}] creates a temporary
with storage class \verb|sclass| and a type synonymous with type suffix \verb|t|
and returns a pointer the symbol-table entry.
If an existing temporary with the appropriate class and type is available,
it is used; otherwise, the new temporary is announced by calling \verb|local|.

\item[{\tt void outflush(void)}] writes the current output buffer
to the standard output, if it's not empty.

\item[{\tt void outs(char *s)}] appends string \verb|s| to the output buffer
for standard output and calls \verb|outflush|
if the resulting buffer pointer is within 80 characters
of the end of the buffer.

\item[{\tt int optype(op)}] is the type suffix of the type-specific operator \verb|op|.

\item[{\tt void print(char *fmt, \ldots)}] prints up to 5 arguments on standard output
similar to \verb|printf|. The supported formats are
\verb|%c|, \verb|%d|, \verb|%o|, \verb|%x|, and \verb|%s|, 
and precision and field width specifications are not supported.
\verb|print| calls \verb|outflush| if it prints a newline character from
\verb|fmt| within 80 characters of the end of the output buffer,
and each format except \verb|%c| does the actual output
with \verb|outs|, which may also flush the buffer.

\item[{\tt int roundup(int n, int m)}] is \verb|n| rounded up to the
next multiple of \verb|m| where \verb|m| is a power of 2.

\item[{\tt char *string(char *s)}] installs \verb|s| in the string space,
if necessary, and returns a pointer to the installed copy.

\item[{\tt char *stringd(int n)}] returns the string representation of \verb|n|;
the returned string is installed in the string space by \verb|string|.

\item[{\tt char *stringf(char *fmt, \ldots)}] formats up to 5 arguments
into a string in the string space and returns a pointer to that string.
See \verb|print| for formatting details.

\item[{\tt void *talloc(int n)}] temporarily allocates \verb|n| bytes and returns
a pointer to the first byte. The storage is released at the end of
the current function.  No alignment is guaranteed.

\item[{\tt int ttob(Type t)}] is the type suffix synonymous with type \verb|t|.

\item[{\tt int variadic(Type t)}] is true if type \verb|t| denotes a variadic function.

\end{description}

\end{document}
